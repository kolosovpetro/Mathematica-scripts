\documentclass[12pt, letterpaper]{amsart}
\usepackage[left=1in,right=1in,bottom=1in,top=1in]{geometry}
\usepackage{amsfonts}
\usepackage{amsmath, amssymb}
\usepackage[font=small,labelfont=bf]{caption}
\usepackage[pdfpagelabels,hyperindex,colorlinks=true,linkcolor=blue,urlcolor=magenta,citecolor=green]{hyperref}
\usepackage{amsthm}
\usepackage{float}
\usepackage{mathrsfs}
\usepackage{colonequals}
\usepackage{xparse}

\newenvironment{myitemize}
{ \begin{itemize}
    \setlength{\itemsep}{4pt}
    \setlength{\parskip}{4pt}
    \setlength{\parsep}{4pt}     }
{ \end{itemize}                  }

\newcommand \power [2]{\langle #1 \rangle^{#2}}
\newcommand \iverson[1][s]{[#1 \ \mathrm{is} \ \mathrm{even}]}
\newcommand \floor [1]{\lfloor #1 \rfloor}
\newcommand \coeffA [3][coeffA]{\texttt{#1}[{\texttt{#2}},{\texttt{#3}}]}
\newcommand \coeffH [4][H]{\texttt{#1}[{\texttt{#2}},{\texttt{#3}},{\texttt{#4}}]}
\newcommand \MacaulayPow [4][MacaulayPow]{\texttt{#1}[{\texttt{#2}},{\texttt{#3}},{\texttt{#4}}]}
\newcommand \PiecewisePow [4][PiecewisePow]{\texttt{#1}[{\texttt{#2}},{\texttt{#3}},{\texttt{#4}}]}
\newcommand \ConvolveSum [4][ConvolveSum]{\texttt{{#1}} [{\texttt{#2}},{\texttt{#3}},{\texttt{#4}}]}
\newcommand \polL [4][L]{\texttt{{#1}} [{\texttt{#2}},{\texttt{#3}},{\texttt{#4}}]}
\newcommand \polP [4][P]{\texttt{{#1}} [{\texttt{#2}},{\texttt{#3}},{\texttt{#4}}]}
\newcommand \coeffX [4][X]{\texttt{#1}[{\texttt{#2}},{\texttt{#3}},{\texttt{#4}}]}
\newcommand \commonPowSum [3][S]{\texttt{#1}[{\texttt{#2}},{\texttt{#3}}]}
\NewDocumentCommand \mynotation { O{P} E{^_}{mb} D(){n} }{\mathbf{#1}^{#2}_{#3}(#4)}
\NewDocumentCommand \convSum { O{C} E{^_}{rn} D(){b} }{\mathbf{#1}^{#2}_{#3}(#4)}
%%\NewDocumentCommand \coeffH { O{H} E{_}{r} D(){k}}{\mathbf{#1}_{#2}(#3)}
%%\NewDocumentCommand \coeffX { O{X} E{_}{r} D(){b}}{\mathbf{#1}_{#2}(#3)}
\NewDocumentCommand \polynomL { O{L} E{_}{m} D(){n,k}}{\mathbf{#1}_{#2}(#3)}



\newtheorem{thm}{Theorem}[section]
\newtheorem{cor}[thm]{Corollary}
\newtheorem{prop}[thm]{Proposition}
\newtheorem{lem}[thm]{Lemma}
\newtheorem{conj}[thm]{Conjecture}
\newtheorem{quest}[thm]{Question}
\newtheorem{ppty}[thm]{Property}
\newtheorem{ppties}[thm]{Properties}
\newtheorem{claim}[thm]{Claim}

\theoremstyle{definition}
\newtheorem{defn}[thm]{Definition}
\newtheorem{defns}[thm]{Definitions}
\newtheorem{con}[thm]{Construction}
\newtheorem{exmp}[thm]{Example}
\newtheorem{exmps}[thm]{Examples}
\newtheorem{notn}[thm]{Notation}
\newtheorem{notns}[thm]{Notations}
\newtheorem{addm}[thm]{Addendum}
\newtheorem{exer}[thm]{Exercise}
\newtheorem{limit}[thm]{Limitation}

\theoremstyle{remark}
\newtheorem{rem}[thm]{Remark}
\newtheorem{rems}[thm]{Remarks}
\newtheorem{warn}[thm]{Warning}
\newtheorem{sch}[thm]{Scholium}

\makeatletter
\let\c@equation\c@thm
\raggedbottom
\makeatother
\numberwithin{equation}{section}
%--------Meta Data: Fill in your info------
\title[main\_definitions.m package documentation]
{\texttt{main\_definitions.m} package documentation}
\email{kolosovp94@gmail.com}
\keywords{Binomial theorem, Convolution, Polynomials, Power function, Multinomial theorem, Binomial coefficient, Multinomial coefficient}
\urladdr{https://kolosovpetro.github.io}
\subjclass[2010]{44A35 (primary), 11C08 (secondary)}
\date{\today}
\hypersetup{
pdftitle={main_definitions.m package documentation},
pdfsubject={Discrete Mathematics, Number Theory, Combinatorics},
pdfauthor={Petro Kolosov},
pdfkeywords={Binomial theorem, Convolution transform, Power function, Polynomials, Multinomial theorem, Binomial Coefficient, Bernoulli number, Discrete convolution, Pascal's triangle, Faulhaber's Formula, Power Sum, Worpitzky Identity}
}
\usepackage{microtype}
\begin{document}
\maketitle
\tableofcontents
\section{Introduction}
This file represents a documentation for \texttt{main\_definitions.m} Mathematica package. To get started proceed to GitHub repository \href{https://github.com/KolosovPetro/research_unit_tests}{\textsf{https://github.com/kolosovpetro/research\_unit\_tests}}, fork it, and find the package \texttt{main\_definitions.m}. This package doesn't have any dependencies on other Mathematica packages. To get started simply install it to your Mathematica by clicking \verb"File -> Install...", click \verb"Source" and choose corresponding file in dropped menu. Then recall the package \verb"main_definitions.m" in Mathematica notebook using the command

\begin{center}
\textsf{Needs["MainDefinitions`"]}
\end{center}

Read also \href{http://support.wolfram.com/kb/5648}{\textsf{http://support.wolfram.com/kb/5648}}.

\section{Functions inside the package \texttt{main\_definitions.m}}
\begin{myitemize}
\item $\coeffA{m}{r}$ returns a real coefficient as
\begin{equation*}
\coeffA{m}{r} \colonequals
\begin{cases}
(2r+1)\binom{2r}{r}, & \mathrm{if } \ r=m \\
(2r+1)\binom{2r}{r} \sum_{d=2r+1}^{m} \coeffA{m}{d} \binom{d}{2r+1} \frac{(-1)^{d-1}}{d-r} B_{2d-2r}, & \mathrm{if } \ 0 \leq r < m \\
0, & \mathrm{if } \ r<0 \ \mathrm{or } \ r>m
\end{cases}
\end{equation*}
\item $\polL{m}{n}{k}$ returns the  polynomial of degree $2m$
\begin{equation*}
\polL{m}{n}{k} \colonequals \sum_{r=0}^{m} \coeffA{m}{r} k^r(n-k)^r
\end{equation*}
\item $\polP{m}{n}{b}$ returns the polynomial of degree $2m+1$
\begin{equation*}
\label{p_definition}
\polP{m}{n}{b} \colonequals \sum_{k=0}^{b-1} \polL{m}{n}{k}
\end{equation*}
\item $\coeffH{m}{t}{b}$ returns a real coefficient defined as
\begin{equation*}
\coeffH{m}{t}{b} \colonequals \sum_{j=t}^{m} \binom{j}{t} \coeffA{m}{j} \frac{(-1)^j}{2j-t+1} \binom{2j-t+1}{b} B_{2j-t+1-b}
\end{equation*}
\item $\coeffX{m}{t}{j}$ returns the polynomial of degree $2m-t$
\begin{equation*}
\coeffX{m}{t}{j} \colonequals (-1)^m \sum_{k=1}^{2m-t+1} \coeffH{m}{t}{k} \cdot j^k
\end{equation*}
\item \commonPowSum{p}{n} returns a common power sum
\begin{equation*}
\commonPowSum{p}{n} \colonequals \sum_{k=0}^{n-1} k^p
\end{equation*}

\item \MacaulayPow{x}{n}{a} returns the powered Macaulay bracket
\begin{equation*}
\MacaulayPow{x}{n}{a}=\power{x-a}{n} \colonequals
\begin{cases}
(x-a)^n, &\quad x\geq a \\
0, &\quad \mathrm{otherwise}.
\end{cases}
\end{equation*}
\item \PiecewisePow{x}{n}{a} gives a piecewise defined power function, involving \texttt{Boole}
\begin{equation*}
\PiecewisePow{x}{n}{a} \colonequals x^n\texttt{Boole}[x\geq a]
\end{equation*}
\item $\ConvolveSum{n}{r}{b}$ returns a convolutional power sum
\begin{equation*}
\label{q_definition}
\ConvolveSum{n}{r}{b} \colonequals \sum_{k=0}^{b-1} k^r(n-k)^r
\end{equation*}
\end{myitemize}

\end{document}
